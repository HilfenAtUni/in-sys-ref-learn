
% Default to the notebook output style

    


% Inherit from the specified cell style.




    
\documentclass[11pt]{article}

    
    
    \usepackage[T1]{fontenc}
    % Nicer default font (+ math font) than Computer Modern for most use cases
    \usepackage{mathpazo}

    % Basic figure setup, for now with no caption control since it's done
    % automatically by Pandoc (which extracts ![](path) syntax from Markdown).
    \usepackage{graphicx}
    % We will generate all images so they have a width \maxwidth. This means
    % that they will get their normal width if they fit onto the page, but
    % are scaled down if they would overflow the margins.
    \makeatletter
    \def\maxwidth{\ifdim\Gin@nat@width>\linewidth\linewidth
    \else\Gin@nat@width\fi}
    \makeatother
    \let\Oldincludegraphics\includegraphics
    % Set max figure width to be 80% of text width, for now hardcoded.
    \renewcommand{\includegraphics}[1]{\Oldincludegraphics[width=.8\maxwidth]{#1}}
    % Ensure that by default, figures have no caption (until we provide a
    % proper Figure object with a Caption API and a way to capture that
    % in the conversion process - todo).
    \usepackage{caption}
    \DeclareCaptionLabelFormat{nolabel}{}
    \captionsetup{labelformat=nolabel}

    \usepackage{adjustbox} % Used to constrain images to a maximum size 
    \usepackage{xcolor} % Allow colors to be defined
    \usepackage{enumerate} % Needed for markdown enumerations to work
    \usepackage{geometry} % Used to adjust the document margins
    \usepackage{amsmath} % Equations
    \usepackage{amssymb} % Equations
    \usepackage{textcomp} % defines textquotesingle
    % Hack from http://tex.stackexchange.com/a/47451/13684:
    \AtBeginDocument{%
        \def\PYZsq{\textquotesingle}% Upright quotes in Pygmentized code
    }
    \usepackage{upquote} % Upright quotes for verbatim code
    \usepackage{eurosym} % defines \euro
    \usepackage[mathletters]{ucs} % Extended unicode (utf-8) support
    \usepackage[utf8x]{inputenc} % Allow utf-8 characters in the tex document
    \usepackage{fancyvrb} % verbatim replacement that allows latex
    \usepackage{grffile} % extends the file name processing of package graphics 
                         % to support a larger range 
    % The hyperref package gives us a pdf with properly built
    % internal navigation ('pdf bookmarks' for the table of contents,
    % internal cross-reference links, web links for URLs, etc.)
    \usepackage{hyperref}
    \usepackage{longtable} % longtable support required by pandoc >1.10
    \usepackage{booktabs}  % table support for pandoc > 1.12.2
    \usepackage[inline]{enumitem} % IRkernel/repr support (it uses the enumerate* environment)
    \usepackage[normalem]{ulem} % ulem is needed to support strikethroughs (\sout)
                                % normalem makes italics be italics, not underlines
    \usepackage{mathrsfs}
    

    
    
    % Colors for the hyperref package
    \definecolor{urlcolor}{rgb}{0,.145,.698}
    \definecolor{linkcolor}{rgb}{.71,0.21,0.01}
    \definecolor{citecolor}{rgb}{.12,.54,.11}

    % ANSI colors
    \definecolor{ansi-black}{HTML}{3E424D}
    \definecolor{ansi-black-intense}{HTML}{282C36}
    \definecolor{ansi-red}{HTML}{E75C58}
    \definecolor{ansi-red-intense}{HTML}{B22B31}
    \definecolor{ansi-green}{HTML}{00A250}
    \definecolor{ansi-green-intense}{HTML}{007427}
    \definecolor{ansi-yellow}{HTML}{DDB62B}
    \definecolor{ansi-yellow-intense}{HTML}{B27D12}
    \definecolor{ansi-blue}{HTML}{208FFB}
    \definecolor{ansi-blue-intense}{HTML}{0065CA}
    \definecolor{ansi-magenta}{HTML}{D160C4}
    \definecolor{ansi-magenta-intense}{HTML}{A03196}
    \definecolor{ansi-cyan}{HTML}{60C6C8}
    \definecolor{ansi-cyan-intense}{HTML}{258F8F}
    \definecolor{ansi-white}{HTML}{C5C1B4}
    \definecolor{ansi-white-intense}{HTML}{A1A6B2}
    \definecolor{ansi-default-inverse-fg}{HTML}{FFFFFF}
    \definecolor{ansi-default-inverse-bg}{HTML}{000000}

    % commands and environments needed by pandoc snippets
    % extracted from the output of `pandoc -s`
    \providecommand{\tightlist}{%
      \setlength{\itemsep}{0pt}\setlength{\parskip}{0pt}}
    \DefineVerbatimEnvironment{Highlighting}{Verbatim}{commandchars=\\\{\}}
    % Add ',fontsize=\small' for more characters per line
    \newenvironment{Shaded}{}{}
    \newcommand{\KeywordTok}[1]{\textcolor[rgb]{0.00,0.44,0.13}{\textbf{{#1}}}}
    \newcommand{\DataTypeTok}[1]{\textcolor[rgb]{0.56,0.13,0.00}{{#1}}}
    \newcommand{\DecValTok}[1]{\textcolor[rgb]{0.25,0.63,0.44}{{#1}}}
    \newcommand{\BaseNTok}[1]{\textcolor[rgb]{0.25,0.63,0.44}{{#1}}}
    \newcommand{\FloatTok}[1]{\textcolor[rgb]{0.25,0.63,0.44}{{#1}}}
    \newcommand{\CharTok}[1]{\textcolor[rgb]{0.25,0.44,0.63}{{#1}}}
    \newcommand{\StringTok}[1]{\textcolor[rgb]{0.25,0.44,0.63}{{#1}}}
    \newcommand{\CommentTok}[1]{\textcolor[rgb]{0.38,0.63,0.69}{\textit{{#1}}}}
    \newcommand{\OtherTok}[1]{\textcolor[rgb]{0.00,0.44,0.13}{{#1}}}
    \newcommand{\AlertTok}[1]{\textcolor[rgb]{1.00,0.00,0.00}{\textbf{{#1}}}}
    \newcommand{\FunctionTok}[1]{\textcolor[rgb]{0.02,0.16,0.49}{{#1}}}
    \newcommand{\RegionMarkerTok}[1]{{#1}}
    \newcommand{\ErrorTok}[1]{\textcolor[rgb]{1.00,0.00,0.00}{\textbf{{#1}}}}
    \newcommand{\NormalTok}[1]{{#1}}
    
    % Additional commands for more recent versions of Pandoc
    \newcommand{\ConstantTok}[1]{\textcolor[rgb]{0.53,0.00,0.00}{{#1}}}
    \newcommand{\SpecialCharTok}[1]{\textcolor[rgb]{0.25,0.44,0.63}{{#1}}}
    \newcommand{\VerbatimStringTok}[1]{\textcolor[rgb]{0.25,0.44,0.63}{{#1}}}
    \newcommand{\SpecialStringTok}[1]{\textcolor[rgb]{0.73,0.40,0.53}{{#1}}}
    \newcommand{\ImportTok}[1]{{#1}}
    \newcommand{\DocumentationTok}[1]{\textcolor[rgb]{0.73,0.13,0.13}{\textit{{#1}}}}
    \newcommand{\AnnotationTok}[1]{\textcolor[rgb]{0.38,0.63,0.69}{\textbf{\textit{{#1}}}}}
    \newcommand{\CommentVarTok}[1]{\textcolor[rgb]{0.38,0.63,0.69}{\textbf{\textit{{#1}}}}}
    \newcommand{\VariableTok}[1]{\textcolor[rgb]{0.10,0.09,0.49}{{#1}}}
    \newcommand{\ControlFlowTok}[1]{\textcolor[rgb]{0.00,0.44,0.13}{\textbf{{#1}}}}
    \newcommand{\OperatorTok}[1]{\textcolor[rgb]{0.40,0.40,0.40}{{#1}}}
    \newcommand{\BuiltInTok}[1]{{#1}}
    \newcommand{\ExtensionTok}[1]{{#1}}
    \newcommand{\PreprocessorTok}[1]{\textcolor[rgb]{0.74,0.48,0.00}{{#1}}}
    \newcommand{\AttributeTok}[1]{\textcolor[rgb]{0.49,0.56,0.16}{{#1}}}
    \newcommand{\InformationTok}[1]{\textcolor[rgb]{0.38,0.63,0.69}{\textbf{\textit{{#1}}}}}
    \newcommand{\WarningTok}[1]{\textcolor[rgb]{0.38,0.63,0.69}{\textbf{\textit{{#1}}}}}
    
    
    % Define a nice break command that doesn't care if a line doesn't already
    % exist.
    \def\br{\hspace*{\fill} \\* }
    % Math Jax compatibility definitions
    \def\gt{>}
    \def\lt{<}
    \let\Oldtex\TeX
    \let\Oldlatex\LaTeX
    \renewcommand{\TeX}{\textrm{\Oldtex}}
    \renewcommand{\LaTeX}{\textrm{\Oldlatex}}
    % Document parameters
    % Document title
    \title{exercise-01}
    
    
    
    
    

    % Pygments definitions
    
\makeatletter
\def\PY@reset{\let\PY@it=\relax \let\PY@bf=\relax%
    \let\PY@ul=\relax \let\PY@tc=\relax%
    \let\PY@bc=\relax \let\PY@ff=\relax}
\def\PY@tok#1{\csname PY@tok@#1\endcsname}
\def\PY@toks#1+{\ifx\relax#1\empty\else%
    \PY@tok{#1}\expandafter\PY@toks\fi}
\def\PY@do#1{\PY@bc{\PY@tc{\PY@ul{%
    \PY@it{\PY@bf{\PY@ff{#1}}}}}}}
\def\PY#1#2{\PY@reset\PY@toks#1+\relax+\PY@do{#2}}

\expandafter\def\csname PY@tok@w\endcsname{\def\PY@tc##1{\textcolor[rgb]{0.73,0.73,0.73}{##1}}}
\expandafter\def\csname PY@tok@c\endcsname{\let\PY@it=\textit\def\PY@tc##1{\textcolor[rgb]{0.25,0.50,0.50}{##1}}}
\expandafter\def\csname PY@tok@cp\endcsname{\def\PY@tc##1{\textcolor[rgb]{0.74,0.48,0.00}{##1}}}
\expandafter\def\csname PY@tok@k\endcsname{\let\PY@bf=\textbf\def\PY@tc##1{\textcolor[rgb]{0.00,0.50,0.00}{##1}}}
\expandafter\def\csname PY@tok@kp\endcsname{\def\PY@tc##1{\textcolor[rgb]{0.00,0.50,0.00}{##1}}}
\expandafter\def\csname PY@tok@kt\endcsname{\def\PY@tc##1{\textcolor[rgb]{0.69,0.00,0.25}{##1}}}
\expandafter\def\csname PY@tok@o\endcsname{\def\PY@tc##1{\textcolor[rgb]{0.40,0.40,0.40}{##1}}}
\expandafter\def\csname PY@tok@ow\endcsname{\let\PY@bf=\textbf\def\PY@tc##1{\textcolor[rgb]{0.67,0.13,1.00}{##1}}}
\expandafter\def\csname PY@tok@nb\endcsname{\def\PY@tc##1{\textcolor[rgb]{0.00,0.50,0.00}{##1}}}
\expandafter\def\csname PY@tok@nf\endcsname{\def\PY@tc##1{\textcolor[rgb]{0.00,0.00,1.00}{##1}}}
\expandafter\def\csname PY@tok@nc\endcsname{\let\PY@bf=\textbf\def\PY@tc##1{\textcolor[rgb]{0.00,0.00,1.00}{##1}}}
\expandafter\def\csname PY@tok@nn\endcsname{\let\PY@bf=\textbf\def\PY@tc##1{\textcolor[rgb]{0.00,0.00,1.00}{##1}}}
\expandafter\def\csname PY@tok@ne\endcsname{\let\PY@bf=\textbf\def\PY@tc##1{\textcolor[rgb]{0.82,0.25,0.23}{##1}}}
\expandafter\def\csname PY@tok@nv\endcsname{\def\PY@tc##1{\textcolor[rgb]{0.10,0.09,0.49}{##1}}}
\expandafter\def\csname PY@tok@no\endcsname{\def\PY@tc##1{\textcolor[rgb]{0.53,0.00,0.00}{##1}}}
\expandafter\def\csname PY@tok@nl\endcsname{\def\PY@tc##1{\textcolor[rgb]{0.63,0.63,0.00}{##1}}}
\expandafter\def\csname PY@tok@ni\endcsname{\let\PY@bf=\textbf\def\PY@tc##1{\textcolor[rgb]{0.60,0.60,0.60}{##1}}}
\expandafter\def\csname PY@tok@na\endcsname{\def\PY@tc##1{\textcolor[rgb]{0.49,0.56,0.16}{##1}}}
\expandafter\def\csname PY@tok@nt\endcsname{\let\PY@bf=\textbf\def\PY@tc##1{\textcolor[rgb]{0.00,0.50,0.00}{##1}}}
\expandafter\def\csname PY@tok@nd\endcsname{\def\PY@tc##1{\textcolor[rgb]{0.67,0.13,1.00}{##1}}}
\expandafter\def\csname PY@tok@s\endcsname{\def\PY@tc##1{\textcolor[rgb]{0.73,0.13,0.13}{##1}}}
\expandafter\def\csname PY@tok@sd\endcsname{\let\PY@it=\textit\def\PY@tc##1{\textcolor[rgb]{0.73,0.13,0.13}{##1}}}
\expandafter\def\csname PY@tok@si\endcsname{\let\PY@bf=\textbf\def\PY@tc##1{\textcolor[rgb]{0.73,0.40,0.53}{##1}}}
\expandafter\def\csname PY@tok@se\endcsname{\let\PY@bf=\textbf\def\PY@tc##1{\textcolor[rgb]{0.73,0.40,0.13}{##1}}}
\expandafter\def\csname PY@tok@sr\endcsname{\def\PY@tc##1{\textcolor[rgb]{0.73,0.40,0.53}{##1}}}
\expandafter\def\csname PY@tok@ss\endcsname{\def\PY@tc##1{\textcolor[rgb]{0.10,0.09,0.49}{##1}}}
\expandafter\def\csname PY@tok@sx\endcsname{\def\PY@tc##1{\textcolor[rgb]{0.00,0.50,0.00}{##1}}}
\expandafter\def\csname PY@tok@m\endcsname{\def\PY@tc##1{\textcolor[rgb]{0.40,0.40,0.40}{##1}}}
\expandafter\def\csname PY@tok@gh\endcsname{\let\PY@bf=\textbf\def\PY@tc##1{\textcolor[rgb]{0.00,0.00,0.50}{##1}}}
\expandafter\def\csname PY@tok@gu\endcsname{\let\PY@bf=\textbf\def\PY@tc##1{\textcolor[rgb]{0.50,0.00,0.50}{##1}}}
\expandafter\def\csname PY@tok@gd\endcsname{\def\PY@tc##1{\textcolor[rgb]{0.63,0.00,0.00}{##1}}}
\expandafter\def\csname PY@tok@gi\endcsname{\def\PY@tc##1{\textcolor[rgb]{0.00,0.63,0.00}{##1}}}
\expandafter\def\csname PY@tok@gr\endcsname{\def\PY@tc##1{\textcolor[rgb]{1.00,0.00,0.00}{##1}}}
\expandafter\def\csname PY@tok@ge\endcsname{\let\PY@it=\textit}
\expandafter\def\csname PY@tok@gs\endcsname{\let\PY@bf=\textbf}
\expandafter\def\csname PY@tok@gp\endcsname{\let\PY@bf=\textbf\def\PY@tc##1{\textcolor[rgb]{0.00,0.00,0.50}{##1}}}
\expandafter\def\csname PY@tok@go\endcsname{\def\PY@tc##1{\textcolor[rgb]{0.53,0.53,0.53}{##1}}}
\expandafter\def\csname PY@tok@gt\endcsname{\def\PY@tc##1{\textcolor[rgb]{0.00,0.27,0.87}{##1}}}
\expandafter\def\csname PY@tok@err\endcsname{\def\PY@bc##1{\setlength{\fboxsep}{0pt}\fcolorbox[rgb]{1.00,0.00,0.00}{1,1,1}{\strut ##1}}}
\expandafter\def\csname PY@tok@kc\endcsname{\let\PY@bf=\textbf\def\PY@tc##1{\textcolor[rgb]{0.00,0.50,0.00}{##1}}}
\expandafter\def\csname PY@tok@kd\endcsname{\let\PY@bf=\textbf\def\PY@tc##1{\textcolor[rgb]{0.00,0.50,0.00}{##1}}}
\expandafter\def\csname PY@tok@kn\endcsname{\let\PY@bf=\textbf\def\PY@tc##1{\textcolor[rgb]{0.00,0.50,0.00}{##1}}}
\expandafter\def\csname PY@tok@kr\endcsname{\let\PY@bf=\textbf\def\PY@tc##1{\textcolor[rgb]{0.00,0.50,0.00}{##1}}}
\expandafter\def\csname PY@tok@bp\endcsname{\def\PY@tc##1{\textcolor[rgb]{0.00,0.50,0.00}{##1}}}
\expandafter\def\csname PY@tok@fm\endcsname{\def\PY@tc##1{\textcolor[rgb]{0.00,0.00,1.00}{##1}}}
\expandafter\def\csname PY@tok@vc\endcsname{\def\PY@tc##1{\textcolor[rgb]{0.10,0.09,0.49}{##1}}}
\expandafter\def\csname PY@tok@vg\endcsname{\def\PY@tc##1{\textcolor[rgb]{0.10,0.09,0.49}{##1}}}
\expandafter\def\csname PY@tok@vi\endcsname{\def\PY@tc##1{\textcolor[rgb]{0.10,0.09,0.49}{##1}}}
\expandafter\def\csname PY@tok@vm\endcsname{\def\PY@tc##1{\textcolor[rgb]{0.10,0.09,0.49}{##1}}}
\expandafter\def\csname PY@tok@sa\endcsname{\def\PY@tc##1{\textcolor[rgb]{0.73,0.13,0.13}{##1}}}
\expandafter\def\csname PY@tok@sb\endcsname{\def\PY@tc##1{\textcolor[rgb]{0.73,0.13,0.13}{##1}}}
\expandafter\def\csname PY@tok@sc\endcsname{\def\PY@tc##1{\textcolor[rgb]{0.73,0.13,0.13}{##1}}}
\expandafter\def\csname PY@tok@dl\endcsname{\def\PY@tc##1{\textcolor[rgb]{0.73,0.13,0.13}{##1}}}
\expandafter\def\csname PY@tok@s2\endcsname{\def\PY@tc##1{\textcolor[rgb]{0.73,0.13,0.13}{##1}}}
\expandafter\def\csname PY@tok@sh\endcsname{\def\PY@tc##1{\textcolor[rgb]{0.73,0.13,0.13}{##1}}}
\expandafter\def\csname PY@tok@s1\endcsname{\def\PY@tc##1{\textcolor[rgb]{0.73,0.13,0.13}{##1}}}
\expandafter\def\csname PY@tok@mb\endcsname{\def\PY@tc##1{\textcolor[rgb]{0.40,0.40,0.40}{##1}}}
\expandafter\def\csname PY@tok@mf\endcsname{\def\PY@tc##1{\textcolor[rgb]{0.40,0.40,0.40}{##1}}}
\expandafter\def\csname PY@tok@mh\endcsname{\def\PY@tc##1{\textcolor[rgb]{0.40,0.40,0.40}{##1}}}
\expandafter\def\csname PY@tok@mi\endcsname{\def\PY@tc##1{\textcolor[rgb]{0.40,0.40,0.40}{##1}}}
\expandafter\def\csname PY@tok@il\endcsname{\def\PY@tc##1{\textcolor[rgb]{0.40,0.40,0.40}{##1}}}
\expandafter\def\csname PY@tok@mo\endcsname{\def\PY@tc##1{\textcolor[rgb]{0.40,0.40,0.40}{##1}}}
\expandafter\def\csname PY@tok@ch\endcsname{\let\PY@it=\textit\def\PY@tc##1{\textcolor[rgb]{0.25,0.50,0.50}{##1}}}
\expandafter\def\csname PY@tok@cm\endcsname{\let\PY@it=\textit\def\PY@tc##1{\textcolor[rgb]{0.25,0.50,0.50}{##1}}}
\expandafter\def\csname PY@tok@cpf\endcsname{\let\PY@it=\textit\def\PY@tc##1{\textcolor[rgb]{0.25,0.50,0.50}{##1}}}
\expandafter\def\csname PY@tok@c1\endcsname{\let\PY@it=\textit\def\PY@tc##1{\textcolor[rgb]{0.25,0.50,0.50}{##1}}}
\expandafter\def\csname PY@tok@cs\endcsname{\let\PY@it=\textit\def\PY@tc##1{\textcolor[rgb]{0.25,0.50,0.50}{##1}}}

\def\PYZbs{\char`\\}
\def\PYZus{\char`\_}
\def\PYZob{\char`\{}
\def\PYZcb{\char`\}}
\def\PYZca{\char`\^}
\def\PYZam{\char`\&}
\def\PYZlt{\char`\<}
\def\PYZgt{\char`\>}
\def\PYZsh{\char`\#}
\def\PYZpc{\char`\%}
\def\PYZdl{\char`\$}
\def\PYZhy{\char`\-}
\def\PYZsq{\char`\'}
\def\PYZdq{\char`\"}
\def\PYZti{\char`\~}
% for compatibility with earlier versions
\def\PYZat{@}
\def\PYZlb{[}
\def\PYZrb{]}
\makeatother


    % Exact colors from NB
    \definecolor{incolor}{rgb}{0.0, 0.0, 0.5}
    \definecolor{outcolor}{rgb}{0.545, 0.0, 0.0}



    
    % Prevent overflowing lines due to hard-to-break entities
    \sloppy 
    % Setup hyperref package
    \hypersetup{
      breaklinks=true,  % so long urls are correctly broken across lines
      colorlinks=true,
      urlcolor=urlcolor,
      linkcolor=linkcolor,
      citecolor=citecolor,
      }
    % Slightly bigger margins than the latex defaults
    
    \geometry{verbose,tmargin=1in,bmargin=1in,lmargin=1in,rmargin=1in}
    
    

    \begin{document}
    
    
    \maketitle
    
    

    
    \begin{longtable}[]{@{}lll@{}}
\toprule
Name & Matriknr. & Studiengang\tabularnewline
\midrule
\endhead
Zhang.Guangde & 4200165 & Msc. Informatik\tabularnewline
Lan.Songnian & 4055640 & Msc. Informatik\tabularnewline
Yi.Zixin & 4126223 & Msc. Informatik\tabularnewline
\bottomrule
\end{longtable}

    \begin{Verbatim}[commandchars=\\\{\}]
{\color{incolor}In [{\color{incolor}1}]:} \PY{k+kn}{import} \PY{n+nn}{numpy} \PY{k}{as} \PY{n+nn}{np}
        \PY{k+kn}{import} \PY{n+nn}{matplotlib}\PY{n+nn}{.}\PY{n+nn}{pyplot} \PY{k}{as} \PY{n+nn}{plt}
        \PY{k+kn}{import} \PY{n+nn}{csv}
        \PY{k+kn}{import} \PY{n+nn}{pandas} \PY{k}{as} \PY{n+nn}{pd}
        \PY{o}{\PYZpc{}}\PY{k}{matplotlib} inline
\end{Verbatim}

    \hypertarget{refresher-on-linear-algebra-and-derivatives}{%
\section{Refresher on Linear Algebra and
Derivatives}\label{refresher-on-linear-algebra-and-derivatives}}

    \begin{itemize}
\item
  \begin{enumerate}
  \def\labelenumi{(\alph{enumi})}
  \tightlist
  \item
    Let \(A\) be a \(3 \times 4\) matrix and \(B\) a \(3 \times 2\)
    matrix, what is the size of \(A^T B\).
  \end{enumerate}
\end{itemize}

    Answer:\\
As we known A is a \(3 \times 4\) matrix, so the transpose of A is a
\(4 \times 3\). Then the size of \(A^T B\) is a \(4 \times 2\) matrix.

\(\because \begin{aligned} A = \left\{\begin{matrix}  a_{11} & a_{12} & a_{13} & a_{14} \\  a_{21} & a_{22} & a_{23} & a_{24} \\  a_{31} & a_{32} & a_{33} & a_{34} \end{matrix}\right\} \end{aligned}\)
and
\(\begin{aligned} B = \left\{\begin{matrix}  b_{11} & b_{12} \\  b_{21} & b_{22} \\  b_{31} & b_{32} \end{matrix}\right\} \end{aligned}\)

\(\therefore \begin{aligned} A^T = \left\{\begin{matrix}  a_{11} & a_{21} & a_{31} \\  a_{12} & a_{22} & a_{32} \\  a_{13} & a_{23} & a_{33} \\  a_{14} & a_{24} & a_{34} \end{matrix}\right\} \end{aligned}\)

\(\begin{aligned} A^TB &= \left\{\begin{matrix}  a_{11} & a_{21} & a_{31} \\  a_{12} & a_{22} & a_{32} \\  a_{13} & a_{23} & a_{33} \\  a_{14} & a_{24} & a_{34} \end{matrix}\right\} \left\{\begin{matrix}  b_{11} & b_{12} \\  b_{21} & b_{22} \\  b_{31} & b_{32} \end{matrix}\right\} \\  &= \left\{\begin{matrix}  \sum_{i=1}^{3}\sum_{j=1}^{3}a_{i1}b_{j1} & \sum_{i=1}^{3}\sum_{j=1}^{3}a_{i1}b_{j2} \\  \sum_{i=2}^{3}\sum_{j=1}^{3}a_{i1}b_{j1} & \sum_{i=2}^{3}\sum_{j=1}^{3}a_{i1}b_{j2} \\  \sum_{i=3}^{3}\sum_{j=1}^{3}a_{i1}b_{j1} & \sum_{i=3}^{3}\sum_{j=1}^{3}a_{i1}b_{j2} \\  \sum_{i=4}^{3}\sum_{j=1}^{3}a_{i1}b_{j1} & \sum_{i=4}^{3}\sum_{j=1}^{3}a_{i1}b_{j2}  \end{matrix}\right\} \end{aligned}\)

    \begin{Verbatim}[commandchars=\\\{\}]
{\color{incolor}In [{\color{incolor}2}]:} \PY{n}{A} \PY{o}{=} \PY{n}{np}\PY{o}{.}\PY{n}{ones}\PY{p}{(}\PY{n}{shape}\PY{o}{=}\PY{p}{(}\PY{l+m+mi}{3}\PY{p}{,}\PY{l+m+mi}{4}\PY{p}{)}\PY{p}{)}
        \PY{n}{B} \PY{o}{=} \PY{n}{np}\PY{o}{.}\PY{n}{ones}\PY{p}{(}\PY{n}{shape}\PY{o}{=}\PY{p}{(}\PY{l+m+mi}{3}\PY{p}{,}\PY{l+m+mi}{2}\PY{p}{)}\PY{p}{)}
        \PY{n}{C} \PY{o}{=} \PY{n}{A}\PY{o}{.}\PY{n}{T}\PY{o}{.}\PY{n}{dot}\PY{p}{(}\PY{n}{B}\PY{p}{)}
        \PY{n+nb}{print}\PY{p}{(}\PY{l+s+s2}{\PYZdq{}}\PY{l+s+s2}{A.T:}\PY{l+s+si}{\PYZob{}\PYZcb{}}\PY{l+s+s2}{ x B:}\PY{l+s+si}{\PYZob{}\PYZcb{}}\PY{l+s+s2}{=C:}\PY{l+s+si}{\PYZob{}\PYZcb{}}\PY{l+s+s2}{\PYZdq{}}\PY{o}{.}\PY{n}{format}\PY{p}{(}\PY{n}{A}\PY{o}{.}\PY{n}{T}\PY{o}{.}\PY{n}{shape}\PY{p}{,} \PY{n}{B}\PY{o}{.}\PY{n}{shape}\PY{p}{,} \PY{n}{C}\PY{o}{.}\PY{n}{shape}\PY{p}{)}\PY{p}{)}
\end{Verbatim}

    \begin{Verbatim}[commandchars=\\\{\}]
A.T:(4, 3) x B:(3, 2)=C:(4, 2)

    \end{Verbatim}

    \begin{itemize}
\item
  \begin{enumerate}
  \def\labelenumi{(\alph{enumi})}
  \setcounter{enumi}{1}
  \tightlist
  \item
    Let \(x \in R^n\) be a column vector (vectors are always columns for
    us) and \(A\) a \(m × n\) matrix. What is the size of \(Ax\).
  \end{enumerate}
\end{itemize}

    Answer:\\
From above we know that , x is a column vector ,also \(n×1\) matrix. A
is a \(m × n\) matrix. So the size of \(Ax\) is \(m×1\) matrix.

assume \(x=\left\{\begin{matrix}x_1 \dots x_n\end{matrix}\right\}^T\)
and
\(\begin{aligned}  A=\left\{\begin{matrix}  A_{1,1} & A_{1,2} & \dots & A_{1,n} \\  A_{2,1} & A_{2,2} & \dots & A_{2,n} \\  \vdots & \vdots & \ddots & \vdots \\  A_{m,1} & A_{m,2} & \dots & A_{m,n}  \end{matrix}\right\} \end{aligned}\)

\(\begin{aligned}  Ax &= \left\{\begin{matrix}  A_{1,1} & A_{1,2} & \dots & A_{1,n} \\  A_{2,1} & A_{2,2} & \dots & A_{2,n} \\  \vdots & \vdots & \ddots & \vdots \\  A_{m,1} & A_{m,2} & \dots & A_{m,n}  \end{matrix}\right\}  \left\{\begin{matrix}x_1 \\  \vdots \\  x_n\end{matrix}\right\} \\  &= \left\{\begin{matrix}  \sum_{i=1}^{n} A_{1,i}x_i \\  \vdots \\  \sum_{i=1}^{n} A_{m,i}x_i \\  \end{matrix}\right\} \end{aligned}\)

    \begin{Verbatim}[commandchars=\\\{\}]
{\color{incolor}In [{\color{incolor}3}]:} \PY{n}{x} \PY{o}{=} \PY{n}{np}\PY{o}{.}\PY{n}{ones}\PY{p}{(}\PY{n}{shape}\PY{o}{=}\PY{p}{(}\PY{l+m+mi}{4}\PY{p}{,}\PY{l+m+mi}{1}\PY{p}{)}\PY{p}{)}
        \PY{n}{A} \PY{o}{=} \PY{n}{np}\PY{o}{.}\PY{n}{ones}\PY{p}{(}\PY{n}{shape}\PY{o}{=}\PY{p}{(}\PY{l+m+mi}{5}\PY{p}{,}\PY{l+m+mi}{4}\PY{p}{)}\PY{p}{)}
        \PY{n}{y} \PY{o}{=} \PY{n}{A}\PY{o}{.}\PY{n}{dot}\PY{p}{(}\PY{n}{x}\PY{p}{)}
        \PY{n+nb}{print}\PY{p}{(}\PY{l+s+s2}{\PYZdq{}}\PY{l+s+s2}{A:}\PY{l+s+si}{\PYZob{}\PYZcb{}}\PY{l+s+s2}{ * x:}\PY{l+s+si}{\PYZob{}\PYZcb{}}\PY{l+s+s2}{ = y:}\PY{l+s+si}{\PYZob{}\PYZcb{}}\PY{l+s+s2}{\PYZdq{}}\PY{o}{.}\PY{n}{format}\PY{p}{(}\PY{n}{A}\PY{o}{.}\PY{n}{shape}\PY{p}{,} \PY{n}{x}\PY{o}{.}\PY{n}{shape}\PY{p}{,} \PY{n}{y}\PY{o}{.}\PY{n}{shape}\PY{p}{)}\PY{p}{)}
\end{Verbatim}

    \begin{Verbatim}[commandchars=\\\{\}]
A:(5, 4) * x:(4, 1) = y:(5, 1)

    \end{Verbatim}

    \begin{itemize}
\item
  \begin{enumerate}
  \def\labelenumi{(\alph{enumi})}
  \setcounter{enumi}{2}
  \tightlist
  \item
    What is the derivative of \(f(x) = (2x + y)^2\) w.r.t.
    x:\(\frac{\partial}{\partial x}f(x)\)
  \end{enumerate}
\end{itemize}

    \[\begin{aligned}
\frac{\partial}{\partial x} f(x) &= \frac{\partial}{\partial x} (2x+y)^2 \\
                                 &=  2(2x+y).2 \\
                                 &= 8x + 4y
\end{aligned}\]

    \begin{itemize}
\item
  \begin{enumerate}
  \def\labelenumi{(\alph{enumi})}
  \setcounter{enumi}{3}
  \tightlist
  \item
    Given \(f(x) = g(x^2)\) where \(g(x) = (x + y)^2\), what is
    \(\frac{\partial}{\partial x}f(x)\)
  \end{enumerate}
\end{itemize}

    \[\begin{aligned}
\frac{\partial}{\partial x}f(x) &= \frac{\partial}{\partial x} g(x^2) \\
                                &= (\frac{\partial}{\partial x} (x^2+y)^2 )*(\frac{\partial}{\partial x} (x^2))\\
                                &= (2(x+y))*(2x) \\
                                &= 4x^3 + 4xy
\end{aligned}\]

    \hypertarget{multivariable-calculus}{%
\section{Multivariable Calculus}\label{multivariable-calculus}}

Recall that a matrix \(A \in R^{n\times n}\) is symmetric if
\(A^T = A\), that is, \(A_{ij} = A_{ji}\) for all \(i, j\). Also recall
the gradient \(\nabla f(x)\) of a function \(f : R^n → R\) is the
\(n\)−vector of partial derivatives

\[
\nabla f(x) = \left\{
\begin{matrix}
    \frac{\partial}{\partial x_1}f(x) \\
    ... \\
    \frac{\partial}{\partial x_1}f(x)
    \end{matrix}
\right\}
\]

where

\[
x = \left\{
\begin{matrix}
    x_1 \\
    ... \\
    x_n
\end{matrix}
\right\}
\]

The hessian \(\nabla^2 f(x)\) is the \(n\times n\) symmetric matrix of
twice partial derivatives,

\[\begin{aligned}
\nabla^2f(x) = \left\{
\begin{matrix}
    \frac{\partial^2}{\partial x_1^2}f(x) & \frac{\partial^2}{\partial x_1 \partial x_2}f(x) & \cdots & \frac{\partial^2}{\partial x_1 \partial x_n} \\
    \frac{\partial^2}{\partial x_2 \partial x_1}f(x) & \frac{\partial^2}{\partial x_2^2}f(x) & \cdots & \frac{\partial^2}{\partial x_2 \partial x_n} \\
    \vdots & \vdots & \ddots & \vdots \\
    \frac{\partial^2}{\partial x_n \partial x_1}f(x) & \frac{\partial^2}{\partial x_n \partial x_2}f(x) & \cdots & \frac{\partial^2}{\partial x_n^2} \\
\end{matrix}
\right\}
\end{aligned}\]

    \begin{itemize}
\item
  \begin{enumerate}
  \def\labelenumi{\alph{enumi})}
  \tightlist
  \item
    Let \(f(x) = \frac{1}{2}x^TAx+b^Tx\), where \(a\) is a sysmmetric
    matrix and \(b\in R^n\) is a vector. What is \(\nabla f(x)\)?
  \end{enumerate}
\end{itemize}

    \(\because\) \(A\in R^{n\times n}\), \(A_{ij} = A_{ji}\) and
\(f(x)= \frac{1}{2}x^TAx+b^Tx\)

\(\therefore\) \[\begin{aligned}
    \frac{\partial}{\partial x_i} f(x) &= \frac{\partial}{\partial x_i} (\frac{1}{2}x^TAx+b^Tx) \\
    &= \frac{\partial}{\partial x_i}(\space \frac{1}{2}\left\{\begin{matrix}
            x_1 \cdots x_n
            \end{matrix}\right\}
            \left\{\begin{matrix}
            A_{1,1} & A_{1,2}\space & \cdots & A_{1,n} \\
            A_{2,1} & A_{2,2}\space & \cdots & A_{2,n} \\
            \vdots & \vdots & \ddots & \vdots \\
            A_{n,1} & A_{n,2} & \cdots & A_{n,n} \\
            \end{matrix}\right\}
            \left\{\begin{matrix}
            x_1 \\
            x_2 \\
            \vdots \\
            x_n
            \end{matrix}\right\}
            +\left\{\begin{matrix}
            b_1\space b_2\space ... b_n
            \end{matrix}\right\}\left\{\begin{matrix}
            x_1 \\
            x_2 \\
            \vdots \\
            x_n
            \end{matrix}\right\} )\\
    &= \frac{\partial}{\partial x_i} (\frac{1}{2} \sum_{i=1}^{n}\sum_{j=1}^{n}A_{ij}x_ix_j + \sum_{j=1}^n b_jx_i) \\
    &= \frac{\partial}{\partial x_i} (\frac{1}{2} (\sum_{i=1}^{n}A_{ii}x_i^2 + \sum_{i=1, i\neq j}^{n}\sum_{j=1, i\neq j}^{n}A_{ij}x_ix_j) + \sum_{j=1}^n b_jx_i) \\
    &= A_{ii}x_i + \sum_{j=1,i\neq j}^{n}A_{ij}x_i + \sum_{j=1}^{n}b_j \\
    &= \sum_{j=1}^{n}A_{ij}x_i + \sum_{j=1}^{n}b_j
\end{aligned}\]

\(\therefore\) \[\begin{aligned}
\nabla f(x) &= \left\{
        \begin{matrix}
            \frac{\partial}{\partial x_1}f(x) \\
            \vdots \\
            \frac{\partial}{\partial x_n}f(x) \\
        \end{matrix}\right\} \\
    &= \left\{
        \begin{matrix}
            \sum_{j=1}^{n}A_{1j}x_1 + \sum_{j=1}^{n}b_j \\
            \vdots \\
            \sum_{j=1}^{n}A_{nj}x_n + \sum_{j=1}^{n}b_j
        \end{matrix}
        \right\} \\
    &= Ax+b
\end{aligned}\]

    \begin{itemize}
\item
  \begin{enumerate}
  \def\labelenumi{\alph{enumi})}
  \setcounter{enumi}{1}
  \tightlist
  \item
    Let \(f(x) = g(h(x))\), where \(g:R\to R\) is differentiable and
    \(h: R^n\to R\) is differentiable. What is \(\nabla f(x)\)?
  \end{enumerate}
\end{itemize}

    \[\begin{aligned}
\nabla f(x) &= \nabla g(h(x)) \\
    &= \frac{\partial g(h(x))}{\partial h(x)} \cdot \frac{\partial h(x)}{\partial x} \\
    &= \frac{\partial g(h(x))}{\partial h(x)} \cdot \left\{\begin{matrix}
        \frac{\partial h(x)}{\partial x_1} \\
        \vdots \\
        \frac{\partial h(x)}{\partial x_i} \\
        \vdots \\
        \frac{\partial h(x)}{\partial x_n}
    \end{matrix}\right\}
\end{aligned}\]

    \begin{itemize}
\item
  \begin{enumerate}
  \def\labelenumi{\alph{enumi})}
  \setcounter{enumi}{2}
  \tightlist
  \item
    Let \(f(x) = \frac{1}{2}x^Tx + b^Tx\) as in a. What is
    \(\nabla^2f(x)\)?
  \end{enumerate}
\end{itemize}

    From (a) knows that \(\nabla f(x) = Ax+b\)

\(\therefore\) \[\begin{aligned}
    \nabla^2f(x) &= \nabla(\nabla f(x)) \\
        &= \nabla(Ax+b) \\
        &= \left\{\begin{matrix}
            \frac{\partial (Ax+b)}{\partial x_1} \\
            \vdots \\
            \frac{\partial (Ax+b)}{\partial x_n}
        \end{matrix}\right\} \\
        &= A
\end{aligned}\]

    \begin{itemize}
\item
  \begin{enumerate}
  \def\labelenumi{\alph{enumi})}
  \setcounter{enumi}{3}
  \tightlist
  \item
    Let \(f(x) = g(a^Tx)\), where \(g:R\to R\) is continuously
    differentiable and \(a\in R^n\) is a vector. What are
    \(\nabla f(x)\) and \(\nabla^2f(x)\)?
  \end{enumerate}
\end{itemize}

    assume that \(h(x) = a^Tx\)

\[\begin{aligned}
\frac{\partial h(x)}{\partial x} &= \frac{\partial a^Tx}{\partial x} \\
        &= a
\end{aligned}\]

\(\therefore\) \[\begin{aligned}
    \nabla f(x) &= \nabla g(h(x)) \\
        &= \frac{\partial g(h)}{\partial h} \cdot \frac{\partial h(x)}{\partial x} \\
        &= \frac{\partial g(h)}{\partial h} \cdot \frac{\partial a^Tx}{\partial x} \\
        &= \frac{\partial g(h)}{\partial h} \cdot a
\end{aligned}\]

\[\begin{aligned}
    \nabla^2 f(x) &= \nabla(\nabla f(x)) \\
        &= \nabla (\frac{\partial g(h)}{\partial h} \cdot a) \\
        &= 0
\end{aligned}\]

    \hypertarget{hands-on}{%
\section{Hands On}\label{hands-on}}

    \begin{Verbatim}[commandchars=\\\{\}]
{\color{incolor}In [{\color{incolor}20}]:} \PY{k}{with} \PY{n+nb}{open}\PY{p}{(}\PY{l+s+s2}{\PYZdq{}}\PY{l+s+s2}{diabetes.txt}\PY{l+s+s2}{\PYZdq{}}\PY{p}{,}\PY{l+s+s2}{\PYZdq{}}\PY{l+s+s2}{r}\PY{l+s+s2}{\PYZdq{}}\PY{p}{)} \PY{k}{as} \PY{n}{file}\PY{p}{:}
             \PY{n}{reader} \PY{o}{=} \PY{n}{csv}\PY{o}{.}\PY{n}{reader}\PY{p}{(}\PY{n}{file}\PY{p}{,} \PY{n}{delimiter}\PY{o}{=}\PY{l+s+s1}{\PYZsq{}}\PY{l+s+s1}{ }\PY{l+s+s1}{\PYZsq{}}\PY{p}{)}
             \PY{n}{table} \PY{o}{=} \PY{n}{np}\PY{o}{.}\PY{n}{asarray}\PY{p}{(}\PY{p}{[}\PY{n}{row} \PY{k}{for} \PY{n}{row} \PY{o+ow}{in} \PY{n}{reader}\PY{p}{]}\PY{p}{,} \PY{n}{dtype}\PY{o}{=}\PY{n}{np}\PY{o}{.}\PY{n}{float}\PY{p}{)}
         
         \PY{c+c1}{\PYZsh{} table.sort(axis=0)}
\end{Verbatim}

    \begin{itemize}
\tightlist
\item
  run the following experiment by taking the first 200 data points as
  training and the next 200 points as validation set
\end{itemize}

    \begin{Verbatim}[commandchars=\\\{\}]
{\color{incolor}In [{\color{incolor}164}]:} \PY{n}{training\PYZus{}set} \PY{o}{=} \PY{n}{table}\PY{p}{[}\PY{p}{:}\PY{l+m+mi}{200}\PY{p}{,}\PY{p}{:}\PY{p}{]} \PY{c+c1}{\PYZsh{}taking the first 200 data points as training set}
          \PY{n}{validation\PYZus{}set} \PY{o}{=} \PY{n}{table}\PY{p}{[}\PY{l+m+mi}{200}\PY{p}{:}\PY{l+m+mi}{400}\PY{p}{,}\PY{p}{:}\PY{p}{]} \PY{c+c1}{\PYZsh{} taking the next 200 points as validation set}
          \PY{c+c1}{\PYZsh{} training set}
          \PY{n}{x\PYZus{}training} \PY{o}{=}\PY{n}{training\PYZus{}set}\PY{p}{[}\PY{p}{:}\PY{p}{,}\PY{p}{:}\PY{l+m+mi}{10}\PY{p}{]}  
          \PY{n}{y\PYZus{}training} \PY{o}{=}\PY{n}{training\PYZus{}set}\PY{p}{[}\PY{p}{:}\PY{p}{,}\PY{l+m+mi}{10}\PY{p}{:}\PY{p}{]} \PY{c+c1}{\PYZsh{}for taining\PYZus{}set choose x\PYZus{}matrix und y\PYZus{}matrix  200hang 1lie }
          \PY{c+c1}{\PYZsh{} validation set}
          \PY{n}{x\PYZus{}validation}\PY{o}{=}\PY{n}{validation\PYZus{}set}\PY{p}{[}\PY{p}{:}\PY{p}{,}\PY{p}{:}\PY{l+m+mi}{10}\PY{p}{]}  
          \PY{n}{y\PYZus{}validation} \PY{o}{=}\PY{n}{validation\PYZus{}set}\PY{p}{[}\PY{p}{:}\PY{p}{,}\PY{l+m+mi}{10}\PY{p}{:}\PY{p}{]}  \PY{c+c1}{\PYZsh{}for validation\PYZus{}set choose x\PYZus{}matrix und y\PYZus{}matrix}
\end{Verbatim}

    \begin{itemize}
\tightlist
\item
  train a least squares regression model without regularization (using
  matrix operations)
\end{itemize}

    \begin{Verbatim}[commandchars=\\\{\}]
{\color{incolor}In [{\color{incolor}165}]:} \PY{c+c1}{\PYZsh{} by solve method}
          \PY{n}{lhs} \PY{o}{=} \PY{n}{np}\PY{o}{.}\PY{n}{dot}\PY{p}{(}\PY{n}{x\PYZus{}training}\PY{o}{.}\PY{n}{T}\PY{p}{,} \PY{n}{x\PYZus{}training}\PY{p}{)}
          \PY{n}{rhs} \PY{o}{=} \PY{n}{np}\PY{o}{.}\PY{n}{dot}\PY{p}{(}\PY{n}{x\PYZus{}training}\PY{o}{.}\PY{n}{T}\PY{p}{,} \PY{n}{y\PYZus{}training}\PY{p}{)}
          \PY{n}{w} \PY{o}{=} \PY{n}{np}\PY{o}{.}\PY{n}{linalg}\PY{o}{.}\PY{n}{solve}\PY{p}{(}\PY{n}{lhs}\PY{p}{,} \PY{n}{rhs}\PY{p}{)}
          \PY{n+nb}{print}\PY{p}{(}\PY{l+s+s2}{\PYZdq{}}\PY{l+s+s2}{solve is: }\PY{l+s+si}{\PYZob{}\PYZcb{}}\PY{l+s+s2}{.T}\PY{l+s+s2}{\PYZdq{}}\PY{o}{.}\PY{n}{format}\PY{p}{(}\PY{n}{w}\PY{o}{.}\PY{n}{T}\PY{p}{)}\PY{p}{)}
          \PY{c+c1}{\PYZsh{} by matrix operate}
          \PY{n}{w} \PY{o}{=} \PY{n}{np}\PY{o}{.}\PY{n}{mat}\PY{p}{(}\PY{n}{x\PYZus{}training}\PY{o}{.}\PY{n}{T}\PY{o}{.}\PY{n}{dot}\PY{p}{(}\PY{n}{x\PYZus{}training}\PY{p}{)}\PY{p}{)}\PY{o}{.}\PY{n}{I}\PY{o}{.}\PY{n}{dot}\PY{p}{(}\PY{n}{x\PYZus{}training}\PY{o}{.}\PY{n}{T}\PY{p}{)}\PY{o}{.}\PY{n}{dot}\PY{p}{(}\PY{n}{y\PYZus{}training}\PY{p}{)}
          \PY{c+c1}{\PYZsh{} print(\PYZdq{}solve is: \PYZob{}\PYZcb{}.T\PYZdq{}.format(w.T))}
\end{Verbatim}

    \begin{Verbatim}[commandchars=\\\{\}]
solve is: [[ -417.97043881  -239.65634372   573.7328081    164.48588506
  -2438.3248339   1673.911313     688.89188963    98.98921971
   1287.00977365   102.9581032 ]].T

    \end{Verbatim}

    \begin{itemize}
\tightlist
\item
  for each regularization parameters
  \(\lambda\in\lbrace2^{-20},2^{-19},...2^{10} \rbrace\) train a ridge
  regression model with regularization  (using matrix operations)
\end{itemize}

    \begin{Verbatim}[commandchars=\\\{\}]
{\color{incolor}In [{\color{incolor}181}]:} \PY{c+c1}{\PYZsh{} parameters \PYZdl{}\PYZbs{}lamda\PYZdl{}}
          \PY{n}{h} \PY{o}{=} \PY{n}{np}\PY{o}{.}\PY{n}{asarray}\PY{p}{(}\PY{p}{[}\PY{n}{i} \PY{k}{for} \PY{n}{i} \PY{o+ow}{in} \PY{n+nb}{range}\PY{p}{(}\PY{o}{\PYZhy{}}\PY{l+m+mi}{20}\PY{p}{,} \PY{l+m+mi}{10}\PY{p}{,} \PY{l+m+mi}{1}\PY{p}{)}\PY{p}{]}\PY{p}{)}
          \PY{n}{lamda} \PY{o}{=} \PY{n}{np}\PY{o}{.}\PY{n}{exp2}\PY{p}{(}\PY{n}{h}\PY{p}{)}
          \PY{n}{d} \PY{o}{=} \PY{l+m+mi}{10}
          \PY{c+c1}{\PYZsh{} calculate w}
          \PY{n}{w} \PY{o}{=} \PY{p}{[}\PY{p}{]}
          \PY{k}{for} \PY{n}{l} \PY{o+ow}{in} \PY{n}{lamda}\PY{p}{:}
              \PY{n}{ww} \PY{o}{=} \PY{n}{np}\PY{o}{.}\PY{n}{linalg}\PY{o}{.}\PY{n}{inv}\PY{p}{(}\PY{n}{np}\PY{o}{.}\PY{n}{sum}\PY{p}{(}\PY{n}{x\PYZus{}training}\PY{o}{.}\PY{n}{T}\PY{o}{.}\PY{n}{dot}\PY{p}{(}\PY{n}{x\PYZus{}training}\PY{p}{)}\PY{p}{)}
                                 \PY{o}{+} \PY{n}{np}\PY{o}{.}\PY{n}{eye}\PY{p}{(}\PY{n}{M}\PY{o}{=}\PY{n}{d}\PY{p}{,}\PY{n}{N}\PY{o}{=}\PY{n}{d}\PY{p}{)}\PY{o}{.}\PY{n}{dot}\PY{p}{(}\PY{n}{l}\PY{p}{)}\PY{p}{)}\PY{o}{.}\PY{n}{dot}\PY{p}{(}\PY{n}{x\PYZus{}training}\PY{o}{.}\PY{n}{T}\PY{p}{)}\PY{o}{.}\PY{n}{dot}\PY{p}{(}\PY{n}{y\PYZus{}training}\PY{p}{)}
              \PY{n}{w}\PY{o}{.}\PY{n}{append}\PY{p}{(}\PY{n}{ww}\PY{p}{)}
          \PY{n}{w} \PY{o}{=} \PY{n}{np}\PY{o}{.}\PY{n}{asarray}\PY{p}{(}\PY{n}{w}\PY{p}{)}
\end{Verbatim}

    \begin{itemize}
\tightlist
\item
  evaluate the trained models on the validation set
\end{itemize}

    \begin{Verbatim}[commandchars=\\\{\}]
{\color{incolor}In [{\color{incolor}186}]:} \PY{n}{err\PYZus{}tra} \PY{o}{=} \PY{p}{[}\PY{p}{]}
          \PY{n}{err\PYZus{}tst} \PY{o}{=} \PY{p}{[}\PY{p}{]}
          \PY{n}{fig} \PY{o}{=} \PY{n}{plt}\PY{o}{.}\PY{n}{figure}\PY{p}{(}\PY{n}{figsize}\PY{o}{=}\PY{p}{(}\PY{l+m+mi}{16}\PY{p}{,}\PY{l+m+mi}{6}\PY{p}{)}\PY{p}{)}
          
          \PY{c+c1}{\PYZsh{} error of training dataset}
          \PY{k}{for} \PY{n}{i}\PY{p}{,} \PY{n}{ww} \PY{o+ow}{in} \PY{n+nb}{enumerate}\PY{p}{(}\PY{n}{w}\PY{p}{)}\PY{p}{:}
              \PY{n}{diff} \PY{o}{=} \PY{n}{np}\PY{o}{.}\PY{n}{subtract}\PY{p}{(}\PY{n}{ww}\PY{o}{.}\PY{n}{T}\PY{o}{.}\PY{n}{dot}\PY{p}{(}\PY{n}{x\PYZus{}training}\PY{o}{.}\PY{n}{T}\PY{p}{)}\PY{p}{,} \PY{n}{y\PYZus{}training}\PY{p}{)}
              \PY{n}{err\PYZus{}tra}\PY{o}{.}\PY{n}{append}\PY{p}{(}\PY{n}{np}\PY{o}{.}\PY{n}{std}\PY{p}{(}\PY{n}{diff}\PY{p}{)}\PY{p}{)}\PY{p}{;}
          \PY{n}{err\PYZus{}tra} \PY{o}{=} \PY{n}{np}\PY{o}{.}\PY{n}{asarray}\PY{p}{(}\PY{n}{err\PYZus{}tra}\PY{p}{)}\PY{p}{;}
          
          \PY{c+c1}{\PYZsh{} error of test dataset}
          \PY{k}{for} \PY{n}{i}\PY{p}{,} \PY{n}{ww} \PY{o+ow}{in} \PY{n+nb}{enumerate}\PY{p}{(}\PY{n}{w}\PY{p}{)}\PY{p}{:}
              \PY{n}{diff} \PY{o}{=} \PY{n}{np}\PY{o}{.}\PY{n}{subtract}\PY{p}{(}\PY{n}{ww}\PY{o}{.}\PY{n}{T}\PY{o}{.}\PY{n}{dot}\PY{p}{(}\PY{n}{x\PYZus{}validation}\PY{o}{.}\PY{n}{T}\PY{p}{)}\PY{p}{,} \PY{n}{y\PYZus{}validation}\PY{p}{)}
              \PY{n}{err\PYZus{}tst}\PY{o}{.}\PY{n}{append}\PY{p}{(}\PY{n}{np}\PY{o}{.}\PY{n}{std}\PY{p}{(}\PY{n}{diff}\PY{p}{)}\PY{p}{)}\PY{p}{;}
          \PY{n}{err\PYZus{}tst} \PY{o}{=} \PY{n}{np}\PY{o}{.}\PY{n}{asarray}\PY{p}{(}\PY{n}{err\PYZus{}tst}\PY{p}{)}\PY{p}{;}
          
          \PY{n}{plt}\PY{o}{.}\PY{n}{plot}\PY{p}{(}\PY{n}{err\PYZus{}tra}\PY{p}{,} \PY{l+s+s1}{\PYZsq{}}\PY{l+s+s1}{r.\PYZhy{}}\PY{l+s+s1}{\PYZsq{}}\PY{p}{)}\PY{p}{;}
          \PY{n}{plt}\PY{o}{.}\PY{n}{plot}\PY{p}{(}\PY{n}{err\PYZus{}tst}\PY{p}{,} \PY{l+s+s1}{\PYZsq{}}\PY{l+s+s1}{b.\PYZhy{}}\PY{l+s+s1}{\PYZsq{}}\PY{p}{)}\PY{p}{;}
          \PY{n}{plt}\PY{o}{.}\PY{n}{grid}\PY{p}{(}\PY{k+kc}{True}\PY{p}{)}\PY{p}{;}
          \PY{n}{plt}\PY{o}{.}\PY{n}{xlabel}\PY{p}{(}\PY{l+s+s1}{\PYZsq{}}\PY{l+s+s1}{\PYZdl{}}\PY{l+s+s1}{\PYZbs{}}\PY{l+s+s1}{lambda\PYZdl{}}\PY{l+s+s1}{\PYZsq{}}\PY{p}{)}\PY{p}{;}
\end{Verbatim}

    \begin{center}
    \adjustimage{max size={0.9\linewidth}{0.9\paperheight}}{output_31_0.png}
    \end{center}
    { \hspace*{\fill} \\}
    
    \begin{itemize}
\tightlist
\item
  plot the results in a style that you fnd most appropriate/informative
  including the selected regularization parameter.
\end{itemize}

    \begin{Verbatim}[commandchars=\\\{\}]
{\color{incolor}In [{\color{incolor}187}]:} \PY{n}{fig} \PY{o}{=} \PY{n}{plt}\PY{o}{.}\PY{n}{figure}\PY{p}{(}\PY{n}{figsize}\PY{o}{=}\PY{p}{(}\PY{l+m+mi}{16}\PY{p}{,}\PY{l+m+mi}{6}\PY{p}{)}\PY{p}{)}
          
          \PY{n}{kb} \PY{o}{=} \PY{p}{[}\PY{p}{]}
          \PY{n}{vkb} \PY{o}{=} \PY{p}{[}\PY{p}{]}
          \PY{n}{oplamda} \PY{o}{=} \PY{l+m+mi}{2}\PY{o}{*}\PY{o}{*}\PY{p}{(}\PY{o}{\PYZhy{}}\PY{l+m+mi}{15}\PY{p}{)}\PY{p}{;}
          \PY{n}{idx} \PY{o}{=} \PY{n+nb}{range}\PY{p}{(}\PY{l+m+mi}{1}\PY{p}{,} \PY{l+m+mi}{201}\PY{p}{,} \PY{l+m+mi}{10}\PY{p}{)}
          
          \PY{k}{for} \PY{n}{n} \PY{o+ow}{in} \PY{n}{idx}\PY{p}{:}
              \PY{n}{x} \PY{o}{=} \PY{n}{x\PYZus{}training}\PY{p}{[}\PY{p}{:}\PY{n}{n}\PY{p}{,} \PY{p}{:}\PY{p}{]}
              \PY{n}{xxt} \PY{o}{=} \PY{n}{x}\PY{o}{.}\PY{n}{dot}\PY{p}{(}\PY{n}{x}\PY{o}{.}\PY{n}{T}\PY{p}{)}
              \PY{n}{xxtl} \PY{o}{=} \PY{n}{x}\PY{o}{.}\PY{n}{dot}\PY{p}{(}\PY{n}{x}\PY{o}{.}\PY{n}{T}\PY{p}{)} \PY{o}{+} \PY{n}{np}\PY{o}{.}\PY{n}{dot}\PY{p}{(}\PY{n}{np}\PY{o}{.}\PY{n}{eye}\PY{p}{(}\PY{n}{M}\PY{o}{=}\PY{n}{n}\PY{p}{,} \PY{n}{N}\PY{o}{=}\PY{n}{n}\PY{p}{)}\PY{p}{,} \PY{n}{oplamda}\PY{p}{)}
          \PY{c+c1}{\PYZsh{}         k.append(np.divide(np.argmax(xxt), np.argmin(xxt)))}
              \PY{n}{kb}\PY{o}{.}\PY{n}{append}\PY{p}{(}\PY{n}{np}\PY{o}{.}\PY{n}{divide}\PY{p}{(}\PY{n}{np}\PY{o}{.}\PY{n}{argmax}\PY{p}{(}\PY{n}{xxtl}\PY{p}{)}\PY{p}{,} \PY{n}{np}\PY{o}{.}\PY{n}{argmin}\PY{p}{(}\PY{n}{xxtl}\PY{p}{)}\PY{p}{)}\PY{p}{)}
              \PY{n}{vx} \PY{o}{=} \PY{n}{x\PYZus{}validation}\PY{p}{[}\PY{p}{:}\PY{n}{n}\PY{p}{,} \PY{p}{:}\PY{p}{]}
              \PY{n}{vxxt} \PY{o}{=} \PY{n}{vx}\PY{o}{.}\PY{n}{dot}\PY{p}{(}\PY{n}{vx}\PY{o}{.}\PY{n}{T}\PY{p}{)}
              \PY{n}{vxxtl} \PY{o}{=} \PY{n}{vx}\PY{o}{.}\PY{n}{dot}\PY{p}{(}\PY{n}{vx}\PY{o}{.}\PY{n}{T}\PY{p}{)} \PY{o}{+} \PY{n}{np}\PY{o}{.}\PY{n}{dot}\PY{p}{(}\PY{n}{np}\PY{o}{.}\PY{n}{eye}\PY{p}{(}\PY{n}{M}\PY{o}{=}\PY{n}{n}\PY{p}{,} \PY{n}{N}\PY{o}{=}\PY{n}{n}\PY{p}{)}\PY{p}{,} \PY{n}{oplamda}\PY{p}{)}\PY{p}{;}
              \PY{n}{vkb}\PY{o}{.}\PY{n}{append}\PY{p}{(}\PY{n}{np}\PY{o}{.}\PY{n}{divide}\PY{p}{(}\PY{n}{np}\PY{o}{.}\PY{n}{argmax}\PY{p}{(}\PY{n}{vxxtl}\PY{p}{)}\PY{p}{,} \PY{n}{np}\PY{o}{.}\PY{n}{argmin}\PY{p}{(}\PY{n}{vxxtl}\PY{p}{)}\PY{p}{)}\PY{p}{)}
          \PY{n}{plt}\PY{o}{.}\PY{n}{plot}\PY{p}{(}\PY{n}{idx}\PY{p}{,} \PY{n}{kb}\PY{p}{,} \PY{l+s+s1}{\PYZsq{}}\PY{l+s+s1}{b.\PYZhy{}}\PY{l+s+s1}{\PYZsq{}}\PY{p}{)}\PY{p}{;}
          \PY{n}{plt}\PY{o}{.}\PY{n}{plot}\PY{p}{(}\PY{n}{idx}\PY{p}{,} \PY{n}{vkb}\PY{p}{,} \PY{l+s+s1}{\PYZsq{}}\PY{l+s+s1}{r.\PYZhy{}}\PY{l+s+s1}{\PYZsq{}}\PY{p}{)}\PY{p}{;}
          \PY{n}{plt}\PY{o}{.}\PY{n}{title}\PY{p}{(}\PY{l+s+s1}{\PYZsq{}}\PY{l+s+s1}{\PYZdl{}}\PY{l+s+s1}{\PYZbs{}}\PY{l+s+s1}{lambda=2\PYZca{}}\PY{l+s+s1}{\PYZob{}}\PY{l+s+s1}{\PYZhy{}15\PYZcb{}\PYZdl{}}\PY{l+s+s1}{\PYZsq{}}\PY{p}{)}\PY{p}{;}
\end{Verbatim}

    \begin{Verbatim}[commandchars=\\\{\}]
/media/self/develop/branch.git/works/uni/publics/runtime/python3.7/lib/python3.7/site-packages/ipykernel\_launcher.py:13: RuntimeWarning: invalid value encountered in true\_divide
  del sys.path[0]
/media/self/develop/branch.git/works/uni/publics/runtime/python3.7/lib/python3.7/site-packages/ipykernel\_launcher.py:17: RuntimeWarning: invalid value encountered in true\_divide

    \end{Verbatim}

    \begin{center}
    \adjustimage{max size={0.9\linewidth}{0.9\paperheight}}{output_33_1.png}
    \end{center}
    { \hspace*{\fill} \\}
    

    % Add a bibliography block to the postdoc
    
    
    
    \end{document}
